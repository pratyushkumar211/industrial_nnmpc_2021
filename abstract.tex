\documentclass[xcolor=dvipsnames, 11pt]{article}

\usepackage[total={4.5in, 8in}]{geometry}
\usepackage[utf8]{inputenc}

\usepackage{graphicx}
\usepackage[dvipsnames]{xcolor}
\usepackage{tikz,tikzsettings,bm}
\usetikzlibrary{shapes,arrows,positioning, calc}
\usepackage{almostfull}
\usepackage{subcaption}

\usepackage{physics}
\usepackage{pgfplots}
\usepackage{ragged2e}

\usepackage{authblk}

%% You should not need more than this for fancy math.
\usepackage{verbatim}
\usepackage{amsmath}   % Extra math commands and environments from the AMS
\usepackage{amssymb}   % Special symbols from the AMS
\usepackage{amsthm}    % Enhanced theorem and proof environments from the AMS
\usepackage{algorithmic}
\usepackage{latexsym}  % A few extra LaTeX symbols
\usepackage{makecell}
%\usepackage{lucidbry}
\input{stanacce}
\renewcommand{\baselinestretch}{1.0666}

%% The URL package is handy for typesetting URLs.  It does not define % an
%\email command because so many document styles already do that. % So we define
%one here that uses a typewriter font.
\usepackage{url}
%\DeclareRobustCommand{\email}{\begingroup \urlstyle{tt}\Url}
\providecommand{\email}{}
\renewcommand{\email}[1]{\texttt{#1}}

%% This provides various customized verbatim commands and % environments.  You
%probably don't need it.
\usepackage{fancyvrb}
\DefineShortVerb{\|} \VerbatimFootnotes
\DefineVerbatimEnvironment{code}{Verbatim}{%
  frame=single, framesep=1em, xleftmargin=1em, xrightmargin=1em, samepage=true,
  fontsize=\footnotesize}
\usepackage{upquote}

\newcommand{\useq}{\mathbf{u}} \newcommand{\xseq}{\mathbf{x}}
\newcommand{\bbR}{\mathbb{R}} \newcommand{\bbU}{\mathbb{U}}
\newcommand{\bbX}{\mathbb{X}}

\usepackage[authoryear,round,longnamesfirst]{natbib}

%% You won't normally need this definition in your documents, but it % is here
%so we can typeset the BibTeX logo correctly.
\makeatletter
\@ifundefined{BibTeX} {\def\BibTeX{{\rmfamily B\kern-.05em%
    \textsc{i\kern-.025em b}\kern-.08em%
    T\kern-.1667em\lower.7ex\hbox{E}\kern-.125emX}}}{}

\def\blfootnote{\xdef\@thefnmark{}\@footnotetext}
\makeatother


\setcounter{topnumber}{2}              %% 2
\setcounter{bottomnumber}{1}           %% 1
\setcounter{totalnumber}{3}            %% 3
\renewcommand{\topfraction}{0.9}       %% 0.7
\renewcommand{\bottomfraction}{0.9}    %% 0.3
\renewcommand{\textfraction}{0.1}      %% 0.2
\renewcommand{\floatpagefraction}{.7}  %% 0.5
\newlength{\graphwidth}
\setlength{\graphwidth}{0.8\columnwidth}

\usepackage{subfiles}

\graphicspath{{./}{./figures/}{./figures/paper/}}

    
\title{Abstract}
\author{Pratyush Kumar, James B. Rawlings, Stephen J. Wright}

\begin{document}

The design of neural networks (NNs) is presented for treating large, linear
model predictive control (MPC) applications that are out of reach with available
quadratic programming (QP) solvers. First, we introduce a new feedforward
network architecture that enables practitioners to obtain offset-free
closed-loop performance with NNs. Second, we discuss the data generation
procedure to sample the state space relevant to training the NNs based on
anticipated online setpoint changes and plant disturbances. Third, we use the
input-to-state stability results available in the MPC literature and establish
robustness properties of NN controllers. Finally, we present illustrative
simulation studies on process control examples. We apply the NN design approach
and compare the performance with online QP based MPC on an industrial crude
distillation unit model with 252 states, 32 control inputs, and a control-sample
horizon length of 140. Parallel computing is used for data generation and
graphical processing units are used for network training. All the anticipated
plant operational scenarios with setpoints and disturbances that may change
during operation must be sampled for NN training. After the offline design
phase, NNs execute MPC three to five orders of magnitude faster than an
available QP solver with less than $1\%$ loss in the closed-loop performance.

\end{document}
